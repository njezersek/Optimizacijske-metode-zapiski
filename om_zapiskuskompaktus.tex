\documentclass[a4paper,8pt]{extarticle}
\usepackage[utf8]{inputenc}

\usepackage{fancyhdr}

\usepackage[pdftex]{graphicx} % Required for including pictures
\usepackage[pdftex,linkcolor=black,pdfborder={0 0 0}]{hyperref} % Format links for pdf
\usepackage{calc} % To reset the counter in the document after title page
\usepackage{enumitem} % Includes lists

\usepackage{textcomp}
\usepackage{eurosym}

\usepackage{ dsfont } % font za množice
% tabele
\usepackage{array}
\usepackage{wrapfig}

\usepackage{tikz,forest}
\usetikzlibrary{arrows.meta}

\frenchspacing % No double spacing between sentences
\setlength{\parindent}{0pt}
\setlength{\parskip}{0.3em}

\usepackage{mathtools}
\usepackage{blkarray, bigstrut} %


\usepackage{amssymb,amsmath,amsthm,amsfonts}
\usepackage{multicol,multirow}
\usepackage{calc}
\usepackage{ifthen}
\usepackage{tabularx}
\usepackage[landscape]{geometry}
\usepackage{listings}
\usepackage{inconsolata}
%\usepackage[colorlinks=true,citecolor=blue,linkcolor=blue]{hyperref}
%\usepackage{accents}

\newcommand{\vect}[1]{\accentset{\rightharpoonup}{#1}}

\ifthenelse{\lengthtest { \paperwidth = 11in}}
    { \geometry{top=.5in,left=.5in,right=.5in,bottom=.5in} }
	{\ifthenelse{ \lengthtest{ \paperwidth = 297mm}}
		{\geometry{top=1cm,left=1cm,right=1cm,bottom=1cm} }
		{\geometry{top=1cm,left=1cm,right=1cm,bottom=1cm} }
	}
\pagestyle{empty}
\makeatletter
\renewcommand{\section}{\@startsection{section}{1}{0mm}%
                                {-1ex plus -.5ex minus -.2ex}%
                                {0.5ex plus .2ex}%x
                                {\normalfont\large\bfseries}}
\renewcommand{\subsection}{\@startsection{subsection}{2}{0mm}%
                                {-1explus -.5ex minus -.2ex}%
                                {0.5ex plus .2ex}%
                                {\normalfont\normalsize\bfseries}}
\renewcommand{\subsubsection}{\@startsection{subsubsection}{3}{0mm}%
                                {-1ex plus -.5ex minus -.2ex}%
                                {1ex plus .2ex}%
                                {\normalfont\small\bfseries}}
\makeatother
\setcounter{secnumdepth}{0}
%\setlength{\parindent}{0pt}
%\setlength{\parskip}{0pt plus 0.5ex}

% listings okolje za psevdo kodo
\lstnewenvironment{koda}[1][] %defines the algorithm listing environment
{   
    \lstset{ %this is the stype
        mathescape=true,
        basicstyle=\scriptsize, 
		columns=flexible,
        keywordstyle=\bfseries\em,
        keywords={,vhod, izhod, zacetek, konec, koncamo, ponavljaj, dokler, ce, vrni, za, vsak, vse, v, sicer,} %add the keywords you want, or load a language as Rubens explains in his comment above.
        xleftmargin=.1\textwidth,
		tabsize=4,
		%frame=leftline,xleftmargin=5pt,xrightmargin=5pt,framesep=5pt,
		%inputencoding = utf8,
		extendedchars = true,
		literate={ž}{{\ˇz}}1 {š}{{\ˇs}}1 {č}{{\ˇc}}1 {Ž}{{\ˇZ}}1 {Š}{{\ˇS}}1 {Č}{{\ˇC}}1,
        #1 % this is to add specific settings to an usage of this environment (for instnce, the caption and referable label)
    }
}
{}
% -----------------------------------------------------------------------

\title{Optimizacijske metode}

\begin{document}

\raggedright
\footnotesize

\begin{multicols}{4}
\setlength{\premulticols}{1pt}
\setlength{\postmulticols}{1pt}
\setlength{\multicolsep}{1pt}
\setlength{\columnsep}{2pt}

\section{Optimizacijske naloge}
Optimizacijska naloga je $(\Omega, f, \text{max/min/sup/inf}, )$ kjer je:
\begin{itemize}
	\item $\Omega$ množica dopustnih rešitev
	\item $f: \Omega \to \mathbb{R}$ kriterijska funkcija
\end{itemize}
$x^* \in \Omega$ je \emph{optimalna} rešitev problema $(\Omega, f, \max)$, če velja
\[\forall x \in \Omega : f(x) \leq f(x^*)\]

\begin{forest} 
	[{\textbf{Optimizacijski problem}}
		[{Nedopustni \\ $\Omega = \emptyset$ }, align=center
		]   
		[{Dopustni \\ $\Omega \neq \emptyset$}, align=center
			[{Neomejen \\ \textit{kriterijska f. neomejena} \\ $\max x$ \\ p.p. $x\geq 0$}, align=center]   
			[Omejen \\ \textit{kriterijska f. omejena}, align=center
				[Optimalni \\ \textit{zaprta meja} \\ $\max x$ \\ p.p. $0 \leq x \leq 1$, align=center]
				[Neoptimalni \\ \textit{odprta meja} \\ $\max x$ \\ p.p. $0 \leq x \leq 1$, align=center]
			]
		]   
	] 
\end{forest}

\section{Linearno programiranje}
$(\Omega, f, \text{min/max})$ je linearni porgram, če je $\Omega$ podana z linearnimi enakostmi in neenakostmi ($\leq$, $\geq$) in je $f$ linearna.

\subsection{Standardna oblika linearnega pograma}
Linearni porgram je v \emph{standardni} obliki, če iščemo $\max$ in so vsi pogoji neenakosti $\leq$ in so vse spremenljivke nenegativne.
\begin{align*}
	\begin{array}{l r@{\;} l} % c@{\hspace{50pt}} r c r@{{}={}} r
		\max 		& c_1 x_1 + \dots + c_n x_n	\\		
		\text{p.p.} & a_{11} x_1 + \dots + a_{1n} x_n &\leq b_1 \\
		& 									& \  \vdots \\										
		& a_{m1} x_1  + \dots + a_{mn} x_n &\leq b_m \\
		& x_1, \dots , x_n & \geq 0
	\end{array}
\end{align*}

To lahko zapišemo v matrični obliki:
\begin{align*}
	c = [c_1 \dots c_n]^T &&
	b = [b_1 \dots b_m]^T &&
	x = [x_1 \dots x_n]^T &&
\end{align*}
\begin{align*}
	A = \begin{bmatrix}
		a_{11} & \dots & a_{1n} \\
		\vdots & 	   & \vdots \\
		a_{m1} & \dots & a_{mn} \\
	\end{bmatrix}
	\in \mathbb{R}^{m \times n}
	&&
	\begin{array}{l@{\hspace{10pt}} c}
		\max & c^Tx \\
		\text{p.p.} & Ax \leq b \\
		& x \geq 0
	\end{array}
\end{align*}

Vsak linearen program lahko zapišemo v standardni obliki.

Vse dele linearnega programa lahko preoblikujemo tako, da bodo v standardni obliki:
\begin{align*}
	\min f(x) &\leadsto \max(-f(x)) \\
	f(x) \geq b &\leadsto -f(x) \leq -b \\
	f(x) = b &\leadsto f(x) \leq b \wedge f(x) \geq b \\
	x_i \leq 0 &\leadsto x_i = -x_i' \\
	x_i \gtrless 0 &\leadsto  x_i = x_i' - x_i'' \wedge x_i', x_i'' \geq 0
\end{align*}


\subsection{Grafično reševanje linearnih programov}
Za linearne porgrame z dvema spremenljivkama lahko narišemo območje, ki ga določajo pogoji. Nato izračunamo gradient kriterijske funkcije in premikamo v smeri gradienta proti točki, ki je v preseku polporostorov pogojev in čim dlje od izhodišča.

\subsection{Simpleksna metoda}
Linearni program zapišemo v standardni obliki. 
Če je kak $b_i < 0$, moramo uporabiti \textbf{dvofazno simpleksno metodo}, sicer nadaljujemo.


Linearni porgram zapišemo v \emph{prvi} slovar.
\begin{gather*}
	\overbrace{
		\begin{array}{r@{\;} l}
			x_{n+1} &= b_1  - a_{11} x_1 - \dots - a_{1n} x_n \\
			& \ \vdots \\
			x_{n+m} &= b_m - a_{m1} x_1 - \dots - a_{mn} x_n \\ \hline
			z &= c_1 x_1 + \dots + c_n x_n
		\end{array}
	}^{\text{1. slovar}}
\end{gather*}

Vse spremenljivke $x_1, \dots, x_{n+m}$ so nenegativne.

Spremenljivke na levi so \textbf{bazne}, na desni pa \textbf{nebazne}.
\begin{align*}
	x_1, \dots, x_n &\quad \dots \quad \textbf{prvotne} \text{ spremenljivke} \\
	x_{n+1}, \dots, x_{n+m} &\quad \dots \quad \textbf{dopolnilne} \text{ spremenljivke}
\end{align*}
Slovar je \textbf{dopusten}, če so vse konstante (čelni brez $x$) na desni nenegativne.

Če je slovar dopusten, ima \emph{bazno} dopustno rešitev: vse \emph{nebazne} spremenljivke so 0 in kriterijska funkcija je tedaj $z = 0$.
\begin{itemize}
	\item Določimo:
	\begin{itemize}
		\item \textbf{vstopno spremenljivko}: izberem spremenljivko, ki ima v kriterijski funkciji pozitiven koeficient. \\
		\item \textbf{pivotno vrstico}: enakost, ki povečanje vstopne spremenljivke najbolj omejuje. Če ni omejena, je problem \textbf{neomejen} in končamo.\\
		\item \textbf{izstopno spremenljivko}: bazna spremenljivka v pivotni vrestici
	\end{itemize}
	\item Iz pivotne vrstice izrazimo vstopno spremenljivko in pivotno vrstico zamenjamo z izražavo (vstopna spremenljivka gre v bazo na levo stran).
	\item V ostalih vrsticah in kriterijski funkciji vstopno spremenljivko nadomestimo z zgornjo izražavo.
	\item Dobimo naslednji slovar. Postopek ponavljamo dokler niso vsi koeficienti v kriterijski funkciji negativni ali enaki 0.
\end{itemize}

Iz zadnjega slovarja razberemo optimalne rešitve: spremenljivke, ki imajo v kriterijski funkiciji negativen koeficient imajo vrednost 0, ostale pa lahko spreminjamo glede na omejitve.

\subsection{Dvofazna simpleksna metoda}
Če je $b \ngeq 0$, uporabimo dvofazno simpleksno metodo.
\subsubsection*{Prva faza}
Konstruiramo pomožni problem. V vsaki neenakosti $b_i$ prištejemo $x_0$. Kriterijsko funkcijo pa spremenimo v $\max -x_0$.

Iz pomožnega problema zapišemo 1. slovar.
\begin{align*}
	\begin{array}{r@{\;} l}
		x_{n+1} &= b_1  + x_0 - a_{11} x_1 - \dots - a_{1n} x_n \\
		& \ \vdots \\
		x_{n+m} &= b_m + x_0 - a_{m1} x_1 - \dots - a_{mn} x_n \\ \hline
		w &= -x_0
	\end{array}
\end{align*}

Za vstopno spremenljivko izberemo $x_0$ za pivotno vrstico pa tisto v kateri je $b_i$ najmanjši. Nato nadaljujemo z običajno simpleksno metodo.

Nadaljujemo z navadno simpleksno metodo in upoštevamo pravilo: $x_0$ ima prednost med kandidati za izstopno spremenljivko.

Če $w* < 0$, prvotni problem ni dopusten, sicer nadlajujemo z drugo fazo.

\subsubsection*{Druga faza}
Iz zadnjega slovarja pomežnega problem izbrišemo $x_0$ in kriterijsko funkcijo originalnega programa izrazimo z nebaznimi spremenljivkami. Nadaljujemo z običajno simpleksno metodo.

\section{Dualnost pri linearnem programiranju}
Vsak linearni program $P$ ima dualno obliko $P'$:

\begin{align*}
	\begin{array}{l@{\hspace{10pt}} c}
		\max & c^Tx \\
		\text{p.p.} & Ax \leq b \\
		& x \geq 0
	\end{array}
	&& \implies &&
	\begin{array}{l@{\hspace{10pt}} c}
		\min & b^Ty \\
		\text{p.p.} & A^Ty \geq c \\
		& y \geq 0
	\end{array}
\end{align*}


	\[P'' = P\]


\subsubsection*{Šibki izrek o dualnosti - ŠID}
$x$ dopustna rešitev za $P$, $y$ dopustna rešitev za $P'$ $\implies$ 
\[c^Tx \leq b^Ty \]

$x$ dopustna za $P$, $y$ dopustna za $P'$ in $c^Tx = b^Ty$ $\implies$
\[	x \textnormal{ optimalna rešitev } P\textnormal{, } y \textnormal{ optimalna rešitev} P' \]

\subsubsection*{Krepki izrek o dualnosti - KID}

$x^*$ optimalna rešitev $P$ $\implies$
\begin{align*}
	\textnormal{ optimalna rešitev } P' \textnormal{ in } c^Tx^* = b^Ty^*
\end{align*}

Linearni program in njegov dual sta lahko:
\begin{tabular}{ r | c | c | c |}
	& nedopusten & neomejen & optimalen \\ \hline
	nedopusten 	& \checkmark & \checkmark & $//_{\text{KID}}$ \\ \hline
	neomejen 	& \checkmark & $//_{\text{ŠID}}$		 & $//_{\text{ŠID, KID}}$ \\ \hline
	optimalen 	& $//_{\text{KID}}$ 		& $//_{\text{ŠID, KID}}$		& \checkmark \\ \hline
\end{tabular}
	
\subsubsection*{Dualno dopolnjevanje}

Naj bo $x$ dopustna za $P$ in $y$ dopustna za $P'$ tedaj je:

$x$ optimalna za $P$ in $y$ optimalna za $P'$ $\iff$
\begin{align*}
	\forall i = 1,\dots , m :\ && \sum_{j=1}^n a_{ij} x_j = b_i \quad &\text{ali} \quad y_i = 0 \\
	\forall j = 1,\dots , n :\ && x_j = 0 \quad &\text{ali} \quad \sum_{i=1}^m a_{ij} y_i = c_j   \\
\end{align*}

\textbf{Ekvivalentno:} $x$ optimalna za $P$, y optimalna za $P'$ $\Leftrightarrow$
\begin{align*}
	\sum_{j=1}^n a_{ij} x_j < b_i &\implies y_i = 0 \quad \forall i \\
	&\ \ \text{in} \\
	x_j > 0 &\implies \sum_{i=1}^m a_{ij} y_i = c_j \quad \forall j \\
\end{align*}

\subsection*{Uporaba izreka o dualnem dopolnjevanju}
Želimo dokazati, da je $x^*$ optimalna rešitev linearnega programa $P$.

\begin{enumerate}
	\item Preverimo, da je $x^*$ dopustna rešitev.
	\item Če je kakšna neenakost pri pogojih $P$ izpolnjena s strogo neenakostjo, je pripadajoča dualna spremenljivka $y^*_i = 0$.
	\item Če je kaka $x^*_j > 0$, je pripadajoča dualna neenakost v $p'$ izpolnjena z enakostjo: \[\sum_{i=1}^n a_{ij} y_i = c_j \]
	\item Vzamemo enačbe iz 3. koraka in upoštevamo, da so nekateri $y$ iz 2. koraka enaki 0. Rešimo dobljeni sistem (če ni rešljiv, $x^*$ ni optimalna).
	\item Preverimo ali je dobljena rešitev $y^*$ dopustna. Če je, sta $x^*$ in $y^*$ optimalni.
\end{enumerate}

\subsection*{Dual splošnega problema}
Splošna oblika linearnega programa je manj stroga standardna oblika.
Dovolimo, da so pogoji postavljeni z $\leq$ lahko pa tudi z $=$. Poleg tega dovolimo, da nekatere spremenljivke niso omejene z $x_j \geq 0$.

Program v splošni obliki izgleda takole:
\begin{align*}
	\begin{array}{l r@{\;} l >{\scriptsize}l}
		\max 		& \sum_{j=1}^n c_j x_j	\\		
		\text{p.p.} & \sum_{j=1}^n a_{ij} x_i 	&\leq b_i	& \forall i=1,\dots , m'\\
					& \sum_{j=1}^n a_{ij} x_i 	&= b_i	& \forall i=m'+1,\dots , m\\
					& x_i 	 					&\geq 0	&\forall i=1,\dots , n'
	\end{array}
\end{align*}

Njegov dual pa je:
\begin{align*}
	\begin{array}{l r@{\;} l >{\scriptsize}l}
		\min 		& \sum_{i=1}^m b_i y_i	\\		
		\text{p.p.} & \sum_{i=1}^m a_{ij} y_i 	&\leq c_j	& \forall j=1,\dots , n'\\
					& \sum_{i=1}^m a_{ij} y_i 	&= c_j	& \forall j=n'+1,\dots , n\\
					& y_i 	 					&\geq 0	&\forall i=1,\dots , m'
	\end{array}
\end{align*}
\begin{align*}
	\text{enakost} &\overset{\text{dual}}{\longleftrightarrow} \text{poljubna spremenljivka} \\
	\text{neenakost} &\overset{\text{dual}}{\longleftrightarrow} \text{nenegativna spremenljivka} 	
\end{align*}

\subsection*{Dualno dopolnjevanje splošnega problema}
$x$ dopustna za $P$\\
$y$ dopustna za $P'$\\

$x$ optimalna za $P$ in $y$ optimalna za $P'$ $\Leftrightarrow$

\begin{align*}
	\forall i = 1,\dots , m':\ && \sum_{j=1}^n a_{ij} x_j = b_i \quad &\text{ali} \quad y_i = 0 \\
	\forall j = 1,\dots , n' :\ && x_j = 0 \quad &\text{ali} \quad \sum_{i=1}^m a_{ij} y_i = c_j   \\
\end{align*}


\subsection*{Ekonomski pomen dualnih spremenljivk}
Naj bo $P$ linearni program, $\exists$ neizrojena optimalna rešitev (v zadnjem slovarju so vse konstante $> 0$). Potem $\exists \varepsilon > 0$, da velja
	\[\Delta z^* = \sum_{i=1}^m y_i^* \Delta b_i\]
	kjer je $y^*$ optimalna rešitev duala, $\Delta z^*$ sprememba optimalne vrednosti, $\Delta b_i$ pa sprememba dense strani pogojev in $|\Delta b_i| < \varepsilon$.


Torej če desni strani pogojev v $P$ prištejemo dovolj majhen $\Delta b$, se optimalna vrednost $z^*$ programa $P$ spremeni za $\Delta z^* = \Delta b^T y^*$. 

$y*$ nam tedaj da ``tržno ceno'' dobrin. Če želimo povečati dobrino $b_i$, se nam dobiček poveča za $b_i y_i^*$. Torej za enoto dobrine $i$ ne smemo plačati več kot $y_i^*$.

\section{Matrične igre}
Igro igrata 2 igralca. Prvi ima $n$, drugi pa $m$ strategij.

\textbf{Plačilna matrika} $A$ ima $n$ vrstic in $m$ stoplcev. Celica v $i$. vrstici in $j$. stolpcu predstavlja znesek, ki ga drugi plača prvemu, če prvi izbere strategijo $i$, drugi pa $j$. (Če je vrednost negativna, privi plača drugemu.)

Igralca igrata po principu najmanjšega tveganja: izbereta strategijo pri kateri v najslabšem primeru izgubita čim manj.

\begin{align*}
	\text{1. igralec: } && \max_i \min_j a_{ij} &=: M_1 \\
	\text{2. igralec: } && \min_j \max_i a_{ij} &=: M_2
\end{align*}

\[ M_1 \leq M_2\]

$(i_0, j_0)$ je \textbf{sedlo} plačilne matrike $A$, če je $a_{i_0 j_0}$ najmanjši v svoji vrstici in največji v svojem stolpcu.

\[ A \text{ ima sedlo } \iff M_1 = M_2 = a_{i_0 j_0}\]

Če ima $A$ sedlo, je $i_0$ optimalna strategija za prvega, $j_0$ pa za drugega igralca. V tem primeru je $(i_0, j_0)$ \emph{Nashevo ravnovesje} in nobenemu igralcu se ne splača spremeniti strategije.

\subsection*{Mešana strategija}
Igralca svoje strategije izbirata nakjučno z verjetnostjo $x_i$ oziroma $y_j$.

\begin{align*}
	x = (x_1, \dots, x_n) && x_1 \geq 0 && x_1 + \dots + x_n = 1 \\
	y = (y_1, \dots, y_n) && y_1 \geq 0 && y_1 + \dots + y_m = 1 
\end{align*}

\textbf{Matematično upanje} je povprečno izplačilo, če bi igralca igrala veliko iger. Vsako celico v plačilni matriki pomnožimo z verjetnostjo, da bo prišlo do tega izida, in vrednosti seštejemo.
\[ \sum_{i=1}^n \sum_{j=1}^m a_{ij} x_i y_j = \sum_{i=1}^n \left( \sum_{j=1}^m a_{ij} y_j \right) x_i = x^T A y\]

Če 1. igralec igra z neko mešano strategijo $x$, je
\[ \min_y x^T A y = \min_{j=1,\dots , m} \sum_{i=1}^n a_{ij} x_i\]
To pomeni, da se 2. igralec lahko na mešano strategijo optimalno brani z neko \textbf{čisto strategijo} $y = (0,\dots,1,\dots,0)$.

\subsection*{Iskanje optimalne strategije}
Upoštevajoč, da se na mešano strategijo nasprotnik lahko brani z čisto strategijo, dobimo optimizacijska problema.
\begin{align*}
	\text{1. igralec išče } && \max_x \min_y x^TAy &= \max_x \min_{j=1,\dots , m} \sum_{i=1}^n a_{ij} x_i\\
	\text{2. igralec išče } && \min_y \max_x x^TAy &= \min_y \max_{i=1,\dots , n} \sum_{j=1}^m a_{ij} y_j
\end{align*}
Problema zapišemo kot linerana programa.

\emph{1. igralec:}
\begin{align*}
	\begin{array}{l >{\displaystyle}r@{\;} l l}
		\max 		& s	\\		
		\text{p.p.} & -\sum_{i=1}^n a_{ij} x_i + s 	&\leq 0	& \forall j=1,\dots , m\\
		& \sum_{i=1}^n x_i  			&= 1	& \\
		& x_i 	 						&\geq 0	&\forall i=1,\dots , n
	\end{array}
\end{align*}
\emph{2. igralec:}
\begin{align*}
	\begin{array}{l >{\displaystyle}r@{\;} l l}
		\min 		& t	\\		
		\text{p.p.} & -\sum_{j=1}^m a_{ij} y_j + t 	&\geq 0	& \forall i=1,\dots , n\\
					& \sum_{j=1}^m y_j  			&= 1	& \\
					& y_j 	 						&\geq 0	&\forall j=1,\dots , m
	\end{array}
\end{align*}

Opazimo, da sta si linearna programa dualna.
Oba problema sta optimalna in imata enako optimalno vrednost. To je \textbf{vrednost}/\textbf{strateško sedlo} igre.

Strategiji $x^*$ in $y^*$ sta optimalni $\iff$ sta dopustni in velja
\[ \min_j \sum_{i=0}^n a_{ij} x_i^* = \max_i \sum_{j=1}^m a_{ij} y_j^*\]

Igra je \textbf{poštena} $\iff$ ima vrednost 0.

Igra je \textbf{simetrična}, če je $A = -A^T$. Tedaj ima vrednost 0 in je poštena.

\subsection*{Poenostavljanje plačilne matrike}
Vektor $x$ \textbf{dominira} $x'$, če je $\forall i : x_i \geq x_i'$.

Če $i$. vrstica dominira $i'$. vrstico v plačilni matriki, lahko $i'$. vrstico odstranimo.\\
Če $j$. stoplec dominira $j'$. stolpec v plačilni matriki, lahko $j$. stolpec odstranimo.

S tem ne spremenimo optimalne vrednosti.

\section{Problem razvoza}
Imamo usmerjen graf $G = (V,E)$. $G$ je povezan kot neusmerjen graf.
\begin{align*}
	b_v \quad &\dots \quad \text{poraba$-$proizvodnja v vozlišču $v \in V$}\\
	c_e \quad &\dots \quad \text{cena povezave $e \in E$} \\
	x_e \quad &\dots \quad \text{količina razvoza na povezavi $e \in E$}
\end{align*}
Poraba mora biti enako velika kot proizvodnja.
\[ \sum_{v\in V} b_v = 0\]

Rešitev problema je vrednost razvoza za vsako povezavo $x_e$. Da je rešitev dopustna mora veljati 
\[\forall e \in E : \ x_e \geq 0 \]
\[\forall v \in V : \ \sum_{\text{konec}(e) = v} x_e - \sum_{\text{začetek}(e) = v} x_e = b_v \]

Kriterijska funkcija je vsota cen, ki jih bomo plačali za razvoz. Seveda jo želimo minimizirati.
\[ \sum_{e \in E} c_e x_e = c^T x\]

Za problem razvoza lahko zapišemo linearni program in njegov dual.
\begin{align*}
	\begin{array}{l >{\displaystyle}r@{\;} l}
		\min 		& c^T x	\\		
		\text{p.p.} & Ax	& = b	 \\
					& x 	 						&\geq 0	
	\end{array} &&
	\begin{array}{l >{\displaystyle}r@{\;} l}
		\max 		& b^T y	\\		
		\text{p.p.} & A^T y	& \leq c	 \\
	\end{array}
\end{align*}
Kjer je $A$ incidenčna matrika
\begin{align*}
	A = \begin{bmatrix}
		a_{ve}
	\end{bmatrix}_{\substack{v \in V \\ e \in E}}
	&&
	a_{ve} = \begin{cases}
		1 & \text{konec}(e) = v \\
		-1 & \text{začetek}(e) = v \\
		0 & \text{sicer}
	\end{cases}
\end{align*}

Rešitvi $x$ in $y$ optimalni $\iff$
\begin{align*}
	\forall ij \in E: \ &&  x_{ij} = 0 \quad &\vee \quad y_j - y_i = c_{ij}
\end{align*}

\subsection*{Simpleksna metoda na omrežjih}
\begin{enumerate}
	\item Poiščemo drevesno dopustno rešitev $x$
	\item Rešimo $y_i + c_{ij} = y_j$ za $ij \in T$ \\
	\emph{Začnemo s poljubnim vozliščem $y_1 = 0$ iz tega lahko izračunamo vrednosti za vsa ostala vozlišča tako, da se premikamo iz začetnega vozlišča in če gremo po pravi smeri, ceno razvoza povečujemo, sicer pa zmanjšujemo} \\
	Če je $y_i + c_{ij} \geq y_j$ za $\forall ij \in E \setminus T$, je $x$ optimalna rešitev - končamo.
	\item Če je $y_i + c_{ij} < y_j$ za kak $ij \in E \setminus T$, je $ij$ \textbf{vstopna povezava}. Dodamo jo v $T$ in dobimo cikel.
	\[ t = \min \{x_e : e \text{ obratna}\}\]
	Na premih povezavah cikla $x$ povečamo za $t$, na obratnih pa pomanjšamo za $t$.\\
	Povezava na kateri je minimum dosežen, je \textbf{izstopna povezava} in jo odstranimo iz drevesa. \\
	Tako dobimo novo vpeto drevo in se vrnemo na korak 2.
\end{enumerate}

Metoda se lahko zacikla. Ciklanju se izognemo tako, da izberemo koren $r \in V$ in za  izstopno povezavo izberemo najbližjo $r$.

\subsection*{Dvofazna simpleksna metoda na omrežjih}
Z njo poiščemo začetno drevesno rešitev oziroma dokažemo, da ne obstaja.

Skonstruiramo pomožen problem tako, da izberemo koren $r \in V$ in originalnemu porblemu dodamo povezave za $\forall v \in V$:
\begin{itemize}
	\item če je $b_v \geq 0$, dodamo umetno povezamo $rv$ (če še ne obstaja), razvoz $x_{rv} = b_v$, cena $c_v = 1$
	\item če je $b_v < 0$, dodamo umetno povezamo $vr$ (če še ne obstaja),  razvoz $x_{rv} = -b_v$, cena $c_v = 1$
\end{itemize}
Cene originalnih povezav nastavimo na 0.

Rešimo pomožen problem razvoza. Če dobimo rešitev s ceno 0 (ne uporablja pomožnih povezav), je originalni problem dopusten in končna drevesna rešitev pomožnega problema je dopustna drevesna rešitev za prvotni problem. 

Če je proizvodnja večja kot poraba, problem ni rešljiv, a lahko dodamo \textbf{smetišče} z zadostno porabo in ga z brezplačnimi povezavami povežemo z vozlišči s proizvodnjo.

\subsubsection*{Celoštevilske rešitve}
	Za problem razvoza z $b_v \in \mathbb{Z}$ velja:
	\begin{itemize}
		\item če obstaja dopustna rešitev, obstaja tudi celoštevilska dopustna rešitev
		\item če obstaja optimalna rešitev, obstaja tudi celoštevilska optimalna rešitev
	\end{itemize}

\subsection*{Königov izrek o plesnih parih}
\textbf{Dvojno stohastična matrika} je matrika $A \in \mathbb{R}^{n\times n}$ za katero velja:
\begin{align*}
	a_{ij} \geq 0 &&
	\forall i : \ \sum_{j=1}^n a_{ij} = 1 &&
	\forall j : \ \sum_{i=1}^m a_{ij} = 1
\end{align*}

\textbf{Permutacijska matrika} je matrika $P \in \{0,1\}^{n\times n}$, ki ima v vsakem stolpcu in vrstici natanko eno 1.

	Naj bo $A$ dvojno stohastična matrika, potem obstaja premutacijska matrika $P$, da velja $p_{ij} > 0 \implies a_{ij} > 0$.

\subsubsection*{Königov izrek o plesnih parih}
	Naj bo $G$ $r$-regularen graf, potem obstaja popolno prirejanje.

\section{Problem razvoza z omejitvami}
Imamo usmerjen graf $G = (V,E)$. $G$ je povezan kot neusmerjen graf.
\begin{align*}
	b_v \in \mathbb{R} \quad \dots \quad &\text{poraba$-$proizvodnja v vozlišču $v \in V$ }\\
	c_e \in \mathbb{R} \quad \dots \quad &\text{cena povezave $e \in E$} \\
	u_e \in [0,\infty] \quad \dots \quad &\text{kapaciteta povezave $e \in E$} \\
	x_e \in [0, u_e] \quad \dots \quad &\text{količina razvoza na povezavi $e \in E$}
\end{align*}

Problem razvoza z omejitvami lahko zapišemo kot linearen program:
\begin{align*}
	\begin{array}{l >{\displaystyle}r@{\;} l}
		\min 		& c^T x	\\		
		\text{p.p.} & Ax	& = b	 \\
					& x 	 						&\leq u	\\
					& x 	 						&\geq 0	
	\end{array}
\end{align*}
\begin{align*}
	x_e = 0 	\quad & \rightarrow \quad \text{prazna povezava} \\
	x_e = u_e 	\quad & \rightarrow \quad \text{nasičena povezava} \\
\end{align*}
Dopustna rešitev $x$ je \textbf{drevesna dopustna rešitev}, če obstaja vpeto drevo $T$, da so vse povezave izven drevesa prazne ali nasičena. 
\subsection*{Postopek reševanja}
\begin{itemize}
	\item Poiščemo začetno dopustno drevesno rešitev $x$ z drevesom $T$
	\item Izračunamo ceno razvoza $y$ za posamezna vozlišča
	\item Poiščemo vstopno povezavo $ij \notin T$, ki ustreza:
	\begin{itemize}
		\item prazna: $x_{ij} = 0$, $y_i + c_{ij} < y_j \implies $ \\
		$ t = \min \left( \{x_e : e \text{ obratna}\} \cup \{u_e-x_e : e \text{ prema}\}\right)$ \\
		na premih povečamo za $t$, na obratnih pomanjšamo za $t$
		\item nasičena: $x_{ij} = u_{ij}$, $y_i + c_{ij} > y_j \implies $ \\
		$ t = \min \left( \{x_e : e \text{ prema}\} \cup \{u_e-x_e : e \text{ obratna}\}\right)$ \\
		na premih pomanjšamo za $t$, na obratnih povečamo za $t$
	\end{itemize}
\end{itemize}

Začetno dopustno drevesno rešitev poiščemo s pomožnim problemom:

Izberemo koren $r \in V$. Za vsako vozlišče $v$:
\begin{itemize}
	\item $b_v < 0$ (proizvodnja): Če že obstaja povezava $vr$ z kapaciteto $u_{vr} \geq -b_v$, nastavimo razvoz na tej povezavi na $b_v$, sicer dodamo povezavo $vr$ z kapaciteto $\infty$ (dovolimo tudi več povezav med vozlišči).
	\item $b_v \geq 0$ (poraba): Če že obstaja povezava $rv$ z kapaciteto $u_{rv} \geq b_v$, nastavimo razvoz na tej povezavi na $-b_v$, sicer dodamo povezavo $rv$ z kapaciteto $\infty$.
\end{itemize}

Umetne (dodane) povezav imajo ceno 1, prvotne pa 0. Prvotni problem je dopusten $\iff$ vrednost pomožnega problema enaka 0.

\pagebreak
\section{Pretoki in prerezi}
\begin{align*}
	G = (V,E) \quad \dots \quad & \text{usmerjen graf}\\
	s,t \in V \quad \dots \quad & \text{začetno in končno vozlišče} \\
	u_e \in [0, \infty) \quad \dots \quad & \text{kapaciteta povezave}
\end{align*}
Iščemo pretok $x_e$, da veljajo Kirchoffovi zakoni in $0 \leq x_e \leq u_e$.
\[ \sum_{\mathclap{\text{konec}(e) = v}} x_e \quad = \quad \sum_{\mathclap{\text{začetek}(e) = v}} x_e \qquad \forall v \in V \setminus \{s,t\} \]
Radi bi \emph{maksimizirali} pretok:
\[\sum_{\mathclap{\text{začetek}(e) = s}} x_e \quad = \quad  \sum_{\mathclap{\text{konec}(e) = t}} x_e \quad = v\]

\subsubsection{Prevedba na problem razvoza}
\begin{align*}
	b_v = 0 & \quad \forall v \in V & c_e = 0 & \quad \forall e \in E \\
\end{align*}
\begin{align*}
	u_e \text{ ostane nespremenjen}
\end{align*}
Dodamo povezavo $ts$ z kapaciteto $u_{ts} = \infty$ in ceno $c_{ts} = -1$.

\subsubsection{Povečujoča pot}
Zaporedje $s = v_0, v_1, \dots, v_k = t$, da $\forall i = 1, \dots, k$ velja:
\begin{gather*}
	v_{i-1} v_i \in E,\ x_{v_{i-1} v_i} < u_{v_{i-1} v_i} \\
	\text{ali} \\
	v_i v_{i-1} \in E,\ x_{v_{i-1} v_i} > 0 \\
\end{gather*} 
Pretok na premih povezavah povečujoče poti povečamo za $\varepsilon$ na obratnih pa pomanjšamo.
\[ \varepsilon = \min\{x_e\ :\ e \text{ obratna}\} \cup \{u_e - x_e \ :\ e \text{ prema}\}\]

\subsection{Prerez}
Podmnožica $C \subseteq V$ je prerez, če velja $s \in C$ in $t \notin C$. Kapaciteta prereza je:
\[ \sum_{\substack{i \in C \\ j \notin C}} u_{ij} \in [0,\infty) \]
Prostornina pretoka $\leq$ kapaciteta prereza.

Če je prostornina pretoka = kapaciteti prerza, je pretok maksimalen in prerez minimalen.

Za problem pretoka velja natanko eno:
\begin{itemize}
	\item \emph{neomejen}: kapaciteta vsakega prereza je $\infty$
	\item \emph{optimalen}: $\exists$ prerez katerega kapaciteta je enaka maksimalnemu pretoku
\end{itemize}

\section{Prirejanja in pokritja}
Naj bo $G = (V, E)$ graf.

$M \subseteq E$ je \textbf{prirejanje}, če $\forall e, f \in M, e \neq f \implies e \cap f = \emptyset$ \\
$P \subseteq V$ je \textbf{pokritje}, če $\forall e \in E\ \exists v \in P:\ v \in E$

$\mu(G)$ = velikost največjega prirejanja \\
$\tau(G)$ = velikost najmanjšega pokritja

$M$ prirejanje, $P$ pokritje $\implies$ $|M| \leq |P|$

Če je $|M| = |P|$, je $M$ največje prirejanje in $P$ najmanjše pokritje in $\mu(G) = \tau(G) = |M| = |P|$.

V splošnem velja le $\mu(G) \leq \tau(G)$, za dvodelne grafe pa $\mu(G) = \tau(G)$.

$e \in E$ je \textbf{vezana}, če $e \in M$, sicer pa je \textbf{prosta}\\
$v \in V$ je \textbf{vezano}, če $\exists e \in M:\ v \in e$, sicer pa je \textbf{prosto}

\textbf{Alternirajoča pot} je pot na kateri se izmenjujejo proste in vezane povezave.

\textbf{Povečujoča pot} je alternirajoča pot, ki se začen in konča v prostem vozlišču.

Če na povečujoči poti zamenjamo proste in vezane povezave, dobimo za 1 večje prirejanje.

$M$ je največje prirejanje $\iff$ ne obstaja povečujoča pot.

\subsection{Madžarska metoda}
$G = (V, E)$ dvodelni graf, $V = X \cup Y$, $M$ prirejanje

$S = \{\text{prosta vozlišča v }X\}\quad $ $T = \emptyset$

Vsak korak:
\[S' = S \cup \left\{\substack{\text{vozlišča v $X$, do katerih lahko iz $T$}\\ \text{pridemo po vezanih povezavah}}\right\} \]
\[T' = T \cup \left\{\substack{\text{vozlišča v $Y$, do katerih lahko iz $S$}\\ \text{pridemo po prostih povezavah}}\right\} \]

Če $T$ vsebuje prosto vozlišče, imamo povečujočo pot, ki jo uporabimo za povečanje prirejanja.

Sicer pa pridemo do koraka kjer je $T' = T$ in $S' = S$. V tem primeru je $M$ največje prirejanje.

\subsubsection{Hallov izrek}
$G = (V, E)$ dvodelni graf, $V = X \cup Y$
\[ \exists\ \text{popolno prirejanje iz $X$ v $Y$} \iff \forall A \subseteq X:\ |A| \leq |N(A)| \]

\subsection{Madžarska metoda z utežmi}
Imamo plon graf $K_{n,n}$; povezava med $x_i$ in $y_j$ ima utež $c_{ij}$
\[ c = [c_{ij}] \in \mathbb{R}^{n\times n}\]
Popolno prirejanje je podano z $\pi \in S_n$: $x_i \sim y_{\pi(i)}$.

Iščemo prirejanje z najmanjšo utežjo:
\[\min_{\pi \in S_n} \sum_{i=1}^n c_{i \pi(i)}\]
\emph{Interpretacija: Razporeditev $n$ opravi $n$ ljudem.}

\subsubsection{Madžarska metoda - postopek}
\begin{enumerate}
	\item Od vsake vrstice odštejemo njen minimum.\\
		Od vsakega stolpca odštejemo njegov minimum.
		\emph{V vsaki vrstici in stoplpcu je vsaj ena ničla}
	\item Pokrijemo vse ničle v matriki pokrijemo z manj kot $n$ vrsticami in stolpci.
		\[ \varepsilon := \text{najmanjše nepokrito polje} > 0\]
		\begin{itemize}
			\item $2\times$ pokritim poljem prištejemo $\varepsilon$
			\item nepokritim pa odštejemo $\varepsilon$
		\end{itemize}
	\item Če ne najdemo takih vrstic in stolpcev, lahko najdemo $n$ ničel v različnih vrsticah in stolpcih. To nam daje minimalno popolno prirejanje.
\end{enumerate}

\section{Iskanje najkrajše poti}
\subsubsection{Pregled v širino (BFS) \small{$O(|V|+|E|)$}}
\begin{koda}[]
vhod: neutezen, neusmerjen graf $G$, zacetno vozlisce $r$
izhod: razdalje med vozliscem $r$ in ostalimi
$Q \leftarrow {r}$
$d(r) \leftarrow 0$
$\pi(r) \leftarrow $ NULL
$\text{obiskan}(r) \leftarrow \text{NE}$
za vsak $v \in V \setminus \{r\}$:
	$d(v) \leftarrow \infty$
	$\pi(v) \leftarrow $ NULL
	$\text{obiskan}(v) \leftarrow \text{NE}$
dokler $Q \neq \emptyset$:
	$v \in Q$
	$Q \leftarrow Q \setminus {v}$
	$\text{obiskan}(v) \leftarrow \text{JA}$
	za vsak $u \in N(v)$:
		ce $\text{obiskan}(u) = \text{NE}$:
			$d(u) \leftarrow d(v) + 1$
			$\pi(u) = v$
			$Q \leftarrow Q \cup \{u\}$
vrni $d$, $\pi$
\end{koda}

\subsubsection{Dijkstrov algoritem}
\begin{koda}
vhod: usmerjen, utezen ($w_e \geq 0$) graf $G = (V,E)$, koren $r$
izhod: razdalje med vozliscem $r$ in ostalimi
$d(r) \leftarrow 0$
$\pi(r) \leftarrow $ NULL
za vsak $v \in V \setminus \{r\}$:
	$d(v) \leftarrow \infty$
	$\pi(v) \leftarrow $ NULL
$Q \leftarrow V$
dokler $Q \neq \emptyset$:
	$v \leftarrow$ element $Q$ z min $d$
	$Q \leftarrow Q \setminus {v}$
	ce $d(v) = \infty$: 
		koncamo
	sicer:
		za vsak $u \in N(v) \cap Q$:
			ce $d(u) > d(v) + w_{vu}$:
				$d(u) \leftarrow d(v) + w_{vu}$
				$\pi(u) \leftarrow v$
vrni $d$, $\pi$
\end{koda}

\subsubsection{Aciklični graf}
\begin{koda}
vhod: utezen, usmerjen graf $G = (V, E)$ brez ciklov
izhod: topoloska urejenost $\varphi$
za vsak $v \in V$:
	$\text{st}(v) \leftarrow \deg^+(v)$
	$i \leftarrow 1$
dokler $\exists v \in V:\ st(v) = 0$:
	$\varphi(v) \leftarrow i$
	za vse $vu \in E$:
		$\text{st}(u) \leftarrow \text{st}(u) - 1$
	$i \leftarrow i + 1$
ce $i \leq |V|$:
	vrni FALSE
sicer:
	vrni $\varphi$
\end{koda}

\begin{koda}
vhod: topoloska urejenost $\varphi$, koren $r$
izhod: razdalje med vozliscem $r$ in ostalimi
$d(r) \leftarrow 0$
$\pi(r) \leftarrow$ NULL
za vsak $v \in V \setminus \{r\}$:
	$d(v) \leftarrow \infty$
	$\pi(v) \leftarrow $ NULL
$i \leftarrow \varphi(r)$
za vsak $j \in \{i, i+1, \dots, |V|\}$:
	$v \leftarrow \varphi^{-1}(j)$
	za vsak $vu \in E$:
		ce $d(u) = d(v) + w_{ue}$:
			$d(u) \leftarrow d(v) + w_{vu}$
			$\pi(u) \leftarrow v$
vrni $d$, $\pi$
\end{koda}

\subsubsection{Bellman-Ford}
\subsubsection{Floyd-Warshellov algoritem}

\section{Konveksna optimizacija}
\subsection{Afine množice}
Množica $A \subseteq \mathbb{R}^n $, $A \neq \emptyset$ je \textbf{afina}, če velja:
\[ \forall x,y \in A \ \forall \lambda \in \mathbb{R} : (1-\lambda)x + \lambda y \in A \]
\emph{Premica med dvema poljibnima točkama iz $A$ mora biti vsebovana v $A$.}\\

\textbf{Afina kombinacija:}
\begin{align*}
	\alpha_1 x_1 + \dots + \alpha_n x_n && \alpha_1 + \dots + \alpha_n = 1
\end{align*}

Naslednje trditve so ekvivalentne:
\begin{itemize}
	\item $A$ je afina
	\item vsaka afina kombinacija vektorjev iz $A$ je v $A$
	\item $A = V + a = \{ v + a\ |\ v \in V \}$\\ za nek $ V \in \mathbb{R}^n \text{ linearen podprostor}$ in $a \in \mathbb{R}^n$
\end{itemize}

\subsection{Konveksne množice}
Množica $A \subseteq \mathbb{R}^n $ je \textbf{konveksna}, če velja:
\[ \forall x,y \in A \ \forall \lambda \in [0,1] : (1-\lambda)x + \lambda y \in A \]
\emph{Daljica med dvema poljibnima točkama iz $A$ mora biti vsebovana v $A$.}\\

Množica \textbf{ni} konveksna, če
\[\exists x,y \in A \ \exists \lambda \in [0, 1]: (1-\lambda)x + \lambda y \notin A \]

\textbf{Konveksna kombinacija:}
\begin{gather*}
	\alpha_1 x_1 + \dots + \alpha_n x_n \\
	\alpha_1 + \dots + \alpha_n = 1 \qquad \alpha_1 + \dots + \alpha_n \geq 0 
\end{gather*}

\emph{Afin podprostor} ( = zaprt za afine kokmbinacije = premaknjen linearen prostor) je \emph{konveksen}.

Presek konveksnih množic $A_i, \forall i\in I$ je konveksen.

Unija koknveksnih množic pa ni nujno konveksna.

$A$ konveksna $\iff$ poljubna konveksna kombinacija vektorjev iz $A$ v $A$.

\textbf{Konveksna ogrinjača:} 
\begin{align*}
	\text{Conv}(A) = \varphi(A) =\ \bigcap_{\mathclap{\substack{K \text{ konv.}\\ A \subseteq K}}}\ K
\end{align*}
\begin{itemize}
	\item $A \subseteq \text{Conv}(A)$
	\item $\text{Conv}(A)$ je konveksna
	\item $A$ konveksna $\implies$ $\text{Conv}(A) = A$
	\item $A \subseteq B$, $B$ konveksna $\implies$ $\text{Conv}(A) \subseteq B$
	\item $\text{Conv}(A) = \{\text{konveksna kombinacija vektorjev iz $A$}\}$
\end{itemize}

\textbf{Ekstremna točka}:
$a \in A^{\text{konv}}$ je ekstremna, če
\begin{align*}
	\forall x,y \in A,\ x,y \neq a:\  \nexists \lambda \in [0,1]:\ a = (1-\lambda)x + \lambda y
\end{align*}
\emph{Točka ni ekstremna, če laži v notranjosti kake daljice med točkama iz $A$.}

Ekstremne točke vedno ležijo na robu. Robne toče pa niso nujno ekstremne.

\subsection{Konveksni stožci}
$A \subseteq \mathbb{R}^n$ je konveksni stožec, če
\[ \forall x,y \in A\ \forall \lambda, \mu \geq 0: \ \lambda x + \mu y \in A \]
Konveksni stožec, je konveksna množica, ni pa nujno obratno.

Vsak linearni podprostor je konveksen stožec.

Konveksni stožec napet na $a_1, \dots, a_n \in \mathbb{R}^n$:
\[S(a_1, \dots, a_n) = \{\lambda_1 a_1 + \dots + \lambda_n a_n\ |\ \lambda_1, \dots, \lambda_n \geq 0\}\]

\textbf{Dualni stožec} množice $A \subseteq \mathbb{R}^n$:
\[ A^* = \left\{ x \in \mathbb{R}^n : x^T a \geq 0\ \forall a \in A \right\}\]

\emph{V $A^*$ so vektorji, ki z vsemi vektorji iz $A$ tvorijo ostri kot.}

\[ A \subseteq A^{**}\]

\subsection{Farkaševa lema}
\textbf{Geometrijska oblika:}
\[ S^{**}(a_1, \dots, a_n) = S(a_1, \dots, a_n)\]
\textbf{Algebraična oblika:}

\emph{Prva varianta:}
\begin{align*}
	\exists x \geq 0:\ Ax = b &\iff \forall y:\ A^Ty \geq 0 \implies b^Ty \geq 0 \\
	b \in S(a_1, \dots, a_n) &\iff b \in S^{**}(a_1, \dots, a_n)
\end{align*}
\emph{Druga varianta:}
\begin{align*}
	\exists x \geq 0:\ Ax \leq b &\iff \forall y \geq 0:\ A^T y \geq 0 \implies b^Ty \geq 0
\end{align*}

\subsection{Konveksne funkcije}
Naj bo $K^\text{konv} \subseteq \mathbb{R}^n$; funkcija $f: K \to \mathbb{R}$ je konveksna, če
\begin{multline*}
	\forall x,y \in K\ \forall \lambda \in [0,1]:\\ f((1-\lambda)x + \lambda y) \leq (1-\lambda)f(x) + \lambda f(y)
\end{multline*}
\emph{Funkcija je konveksna, če je definirana na konveksnem območju in graf vedno leži pod zveznico dveh točk na grafu.}

Če zgoraj velja stroga neenakost, je funkcija \textbf{strogo konveksna}.

$f$ strogo konveksna $\iff$ $H_f > 0$.

Če ima strogo konveksna funkcija maksimum, je v \emph{ekstremni točki}.

\subsection{Konkavne funkcije}
Naj bo $K^\text{konv} \subseteq \mathbb{R}^n$; funkcija $f: K \to \mathbb{R}$ je konkavna, če
\begin{multline*} 
	\forall x,y \in K\ \forall \lambda \in [0,1]:\\ f((1-\lambda)x + \lambda y) \geq (1-\lambda)f(x) + \lambda f(y)
\end{multline*}
\noindent\rule{2cm}{0.4pt}


Če je $f$ konveksna, je $-f$ konkavna.

Funkcija $f$ je \textbf{afina} $\iff$ konveksna in konkavna

\begin{itemize}
	\item $f: K \to \mathbb{R}$, $c \geq 0$, $f$ konveksna $\implies$ $c\cdot f$ konveksna
	\item $f,g: K \to \mathbb{R}$, $f,g$ konveksni $\implies$ $f + g$ konveksna
	\item $g: K \to \mathbb{R}^n$ afina $\implies$ $g(K)$ konveksna\\
		$f: g(K) \to \mathbb{R}$ konveksna $\implies$ $f \circ g$ konveksna
	\item $g: K \to \mathbb{R}$, $f: \text{Conv}(g(K)) \to \mathbb{R}$, $f,g$ konveksni\\
		$f$ naraščujoča $\implies$ $f \circ g$ konveksna
\end{itemize}

\subsection{Konveksne funkcije in optimizacija}
Naj bo $A \subseteq \mathbb{R}^n$ in $f: A \to \mathbb{R}$.
\begin{itemize}
	\item $f$ ima v $x \in A$ \textbf{globalni maksimum}, če
	\[ \forall x' \in A:\ f(x) \geq f(x')\]
	\item $f$ ima v $x \in A$ \textbf{lokalni maksimum}, če
	\[ \exists \varepsilon > 0:\ \forall x' \in A \cap K_\varepsilon (x):\ f(x) \geq f(x')\]
\end{itemize}
Če je $x$ lokalni minimum konveksne funkcije, je tudi globalni minimum.

\subsubsection{Preverjanje konveksnosti funkcije}
\textbf{S prvim odvodom}

$f: K^{\text{konv, odp}} \to \mathbb{R}$ odvedlijva
\[ f \text{ konveksna} \iff f(y) \geq f(x) + \left(\nabla f(x)\right)^T(y-x)\]

\textbf{Z drugim odvodom}

$f: \Omega^{\text{odp}} \subseteq \mathbb{R}^n \to \mathbb{R}$ dvakrat zvezno odvedljiva

Hessejeva matrika:
\[H_f = \begin{bmatrix}
	\frac{\partial^2 f}{\partial x_i \partial x_j}
\end{bmatrix}_{i,j = 1}^n\]

$f$ konveksna $\iff$ $H_f \geq 0$

$f: \Omega \to \mathbb{R}$ konveksna $\iff$ $h_{x,y}:I_{x,y} \to \mathbb{R}$, $h_{x,y}(\lambda) = f(x+\lambda y)$, konveksna za $\forall x \in \mathbb{R}\ \forall y \in \mathbb{R}^n$\\
$I_{x,y} = \{\lambda \in \mathbb{R} : x+\lambda y \in \Omega\}$

\noindent\rule{2cm}{0.4pt}

\textbf{Lastne vrednosti} matrike $A$ so ničle karakterističnega polinoma $det(A - \lambda I)$.

\textbf{Lastni podprostor} vrednosti $\lambda$ je $\text{Ker}(A-\lambda I) \setminus \{0\}$.

$A \in \mathbb{R}^{n\times n}$ je \textbf{diagonalizabilna}, če ima $n$ linearno neodvisnih lastnih vektorjev. Tedaj je $A = PDP^{-1}$, kjer je $D$ diagonalna z lastnimi vrednostmi na diagonali, $P$ pa obrnljiva z lastnimi vektorji v stolpcih.

Če je $A$ \textbf{simetrična} ($A^T = A$), so vse lastne vrednosti realne in $A$ je diagonalizabilna v ortonormirani bazi (lastni vektorji različnih lastnih vrednosti so $\perp$).

Matrika $A$ je \textbf{pozitivno semidefinitna} ($A \geq 0$) \\
$\iff$ vse lastne vrednosti $\lambda \geq 0$ \\
$\iff$ $x^T A x = \sum_{i,j = 1}^n \geq 0$ \\
$\iff$ vse podteterminante $\geq 0$ \\

Matrika $A = \begin{bmatrix}
	a & b \\
	c & d
\end{bmatrix}$ je $A \geq 0$ $\iff$ $ac - b^2 \geq 0$ in $a \geq 0$
\subsection{Konveksne funkcije in vezani ekstremi}
\section{Celoštevilski linearni programi}
\end{multicols}
\end{document}